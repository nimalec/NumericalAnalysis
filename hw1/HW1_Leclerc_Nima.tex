\documentclass[paper=a4, fontsize=11pt]{scrartcl} % A4 paper and 11pt font size

\usepackage[T1]{fontenc} % Use 8-bit encoding that has 256 glyphs
\usepackage{fourier} % Use the Adobe Utopia font for the document - comment this line to return to the LaTeX default
\usepackage[english]{babel} % English language/hyphenation
\usepackage{amsmath,amsfonts,amsthm} % Math packages

\usepackage{lipsum} % Used for inserting dummy 'Lorem ipsum' text into the template

\usepackage{sectsty} % Allows customizing section commands
\allsectionsfont{\centering \normalfont\scshape} % Make all sections centered, the default font and small caps

\usepackage{fancyhdr} % Custom headers and footers
\pagestyle{fancyplain} % Makes all pages in the document conform to the custom headers and footers
\fancyhead{} % No page header - if you want one, create it in the same way as the footers below
\fancyfoot[L]{} % Empty left footer
\fancyfoot[C]{} % Empty center footer
\fancyfoot[R]{\thepage} % Page numbering for right footer
\renewcommand{\headrulewidth}{0pt} % Remove header underlines
\renewcommand{\footrulewidth}{0pt} % Remove footer underlines
\setlength{\headheight}{13.6pt} % Customize the height of the header

\numberwithin{equation}{section} % Number equations within sections (i.e. 1.1, 1.2, 2.1, 2.2 instead of 1, 2, 3, 4)
\numberwithin{figure}{section} % Number figures within sections (i.e. 1.1, 1.2, 2.1, 2.2 instead of 1, 2, 3, 4)
\numberwithin{table}{section} % Number tables within sections (i.e. 1.1, 1.2, 2.1, 2.2 instead of 1, 2, 3, 4)

\setlength\parindent{0pt} % Removes all indentation from paragraphs - comment this line for an assignment with lots of text



\newcommand{\horrule}[1]{\rule{\linewidth}{#1}} % Create horizontal rule command with 1 argument of height

\title{	
\normalfont \normalsize 
\textsc{College of Engineering, Cornell University} \\ [25pt] % Your university, school and/or department name(s)
\horrule{0.5pt} \\[0.4cm] % Thin top horizontal rule
\huge Problem Set 1 , CS 4220\\ % The assignment title
\horrule{2pt} \\[0.5cm] % Thick bottom horizontal rule
}

\author{Nima Leclerc} % Your name

\date{\normalsize\today} % Today's date or a custom date

\begin{document}

\maketitle % Print the title

%----------------------------------------------------------------------------------------
%	PROBLEM 1
%----------------------------------------------------------------------------------------


\section{Question 1: Suppose A;B 2 Rnn; be general square matrices, D 2 Rnn is a diagonal matrix, and u; v 2 Rn
are vectors of length n: For the following mathematical expressions: determine an optimal way
to compute them (in terms of complexity in n), give the complexity and explain your reasoning,
implement your procedure, demonstrate that your code scales correctly by timing your code as
n is increased and providing a plot (think carefully about how to scale the axes). You may reorder,
modify the expressions in any way you choose, so long as the result remains mathematically
equivalent to the given statement.}





\section{Question 2: Say that we would like to compute e􀀀x for x > 0 given only the four basic arithmetic operations.
A natural method to consider would be via the Taylor series of the exponential truncated to some
number of terms and evaluated at 􀀀x. Try this, does it seem to work? Why might this not be such
a good idea? In particular, for x = 20 are you able to accurately compute e􀀀20? You may compare
with the built in function for computing the exponential in MATLAB or Julia (they use schemes
to compute the exponential that are outside the scope of this course, and for the purpose of this
assignment you can consider them as the ground truth).
Given the leading questions in the preceding part of the question, you may have surmised that
nave application of the Taylor series is problematic. However, there is still a way to use Taylor
series to compute the desired quantity, albeit requiring extra arithmetic operations. How might
you accomplish this? Implement and test your method for x = 20 and show you can compute e􀀀20
to at least 10􀀀8 relative error.} 



\end{document}