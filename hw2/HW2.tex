\documentclass{article} 
\usepackage[letterpaper, margin=1in]{geometry} 
\usepackage{listings}
\usepackage{color}
\usepackage{graphicx}
\usepackage{amsmath}
\graphicspath{ Users/nimalec/Documents/SPRING 2018 COURSES-CORNELL/CS 4220/PSETS} 
\lstset{frame=tb,
  language= Matlab,
  aboveskip=3mm,
  belowskip=3mm,
  showstringspaces=false,
  columns=flexible,
  basicstyle={\small\ttfamily},
  numbers=none,
  numberstyle=\tiny\color{gray}
  keywordstyle=\color{blue},
  breaklines=true,
  breakatwhitespace=true, 
  tabsize=3
}


\begin{document} 
\title{\bf{Homework 2}}

\author{Nima Leclerc\\Cornell University\\CS 4220 \\ Numerical Analysis: Linear and Nonlinear Problems}
\date{February 21, 2018}
\maketitle
\newpage   
\subsection*{Question 1:}
When discussing Gaussian elimination and then $LU$ decompositions in class, we considered matrices $G^{(k)}$ that when applied to a matrix $A$ introduced zeros in the $k^{th}$ row below the diagonal. Assuming all of the matrices involved in this problem are $n \times n,$ we may write these matrices compactly as 
\[
G^{(k)} = I - \ell^{(k)}e_k^T,
\]
where $\ell^{(k)} = \left[0,\ldots,0, \ell_{k+1,k},\ldots \ell_{n,k} \right]^T$ is zero in the first $k$ entries. We will now prove two properties of these matrices outlined in class.

\emph{(a)  Prove that} $\left(G^{(k)}\right)^{-1} = I + \ell^{(k)}e_k^T$ \newline \newline 
To show this, it must be proven that $\left(G^{(k)}\right)^{-1}  \left(G^{(k)}\right) =I$. Hence, 
$$ \left(G^{(k)}\right)^{-1}  \left(G^{(k)}\right) = (I + \ell^{(k)}e_k^T)( I - \ell^{(k)}e_k^T) = I^{2} -I\ell^{(k)}e_k^T + \ell^{(k)}e_k^T I - \ell^{(k)}e_k^T\ell^{(k)}e_k^T =   I -\ell^{(k)}e_k^T + \ell^{(k)}e_k^T  - \ell^{(k)}e_k^T\ell^{(k)}e_k^T 
 $$ 
$$ =  I - \ell^{(k)}e_k^T\ell^{(k)}e_k^T = I - (\ell^{(k)}e_k^T)^{2}$$   
Hence, the second term in the right and side of the equation above must be 0. Now expanding this product would require only multiplying the non-zero entry of $e^{k}$  with zero entries of  $\ell^{(k)}$, and vice versa. Hence, this would result in the $n \times n$ zero matrix. 
Hence, this results in $\left(G^{(k)}\right)^{-1} \left(G^{(k)}\right) = I $. This proves that   $\left(G^{(k)}\right)^{-1} = I + \ell^{(k)}e_{k}^T$.  Q.E.D. 
\newline 

\emph{ (b) 
 Prove that} $\displaystyle \left(G^{(1)}\right)^{-1}\left(G^{(2)}\right)^{-1}\cdots\left(G^{(n-1)}\right)^{-1} = I + \sum_{k=1}^{n-1} \ell^{(k)}e_k^T$
\newline 
Here, it requires to be shown that, 
$$ 
\prod_{k=1}^{n-1}   \left(G^{(k)}\right)^{-1} =  I + \sum_{k=1}^{n-1} \ell^{(k)}e_k^T     
$$ 
We can demonstrate this by expanding out to to $n=4$, 
$$ \prod_{k=1}^{4-1}     \left(G^{(k)}\right)^{-1} =   $$ 
$$ \left(G^{(1)}\right)^{-1}   \left(G^{(2)}\right)^{-1}   \left(G^{(3)}\right)^{-1} = $$ 
$$  (I + \ell^{(1)}e_1^T ) (I + \ell^{(2)}e_2^T ) (I + \ell^{(3)}e_3^T )  =$$ 
$$ (I^{2} + I \ell^{(2)}e_2^T  + \ell^{(1)}e_1^T I + \ell^{(1)}e_1^T  \ell^{(2)}e_2^T)( I + \ell^{(3)}e_3^T ) =$$ 
$$  I^{2} +  I^{2} \ell^{(1)}e_1^T+  I^{2} \ell^{(2)}e_2^T +  I \ell^{(1)}e_1^T \ell^{(2)}e_2^T +  I \ell^{(3)}e_3^T  +    I \ell^{(2)}e_2^T \ell^{(3)}e_3^T +      \ell^{(1)}e_1^T \ell^{(2)}e_2^T \ell^{(3)}e_3^T
$$ 
Grouping the terms together, this follows the trend 
$$  \prod_{k=1}^{3}  \left(G^{(k)}\right)^{-1} = I + \sum_{k=1}^{3} \ell^{(k)}e_k^T + \sum_{j=1}^{3}\sum_{k=2}^{3} \ell^{(j)}e_j^T \ell^{(k)}e_k^T $$ 
The higher order terms: $ I \ell^{(1)}e_1^T \ell^{(2)}e_2^T,  I \ell^{(2)}e_2^T \ell^{(3)}e_3^T ,  \ell^{(1)}e_1^T \ell^{(2)}e_2^T \ell^{(3)}e_3^T$ drop to zero, since these products of matrix elements between matrices $\ell^{(i)}e_i^T$ and  $\ell^{(j)}e_j^T$ are ones of products of either zero with zero entries, or zero with non-zero entries. Collectively, this results in a zero matrix. All these cross terms correspond to the double sum  in the above expression. Hence, this double sum evaluates to a zero matrix and the product becomes, 
$$  \prod_{k=1}^{3}  \left(G^{(k)}\right)^{-1} = I + \sum_{k=1}^{3} \ell^{(k)}e_k^T $$ 
\newline 
This can be generalized to to a product of $n-1$ of these $G$ matrix inverses. Hence, one can say 
$$ 
\prod_{k=1}^{n-1}   \left(G^{(k)}\right)^{-1} =  I + \sum_{k=1}^{n-1} \ell^{(k)}e_k^T     
$$  
Q.E.D. 
 
\subsection*{Question 2:}
Assume that you are given an $n \times n$ non-singular matrix of the form 
\[
A = D + ue_n^T + e_nv^T
\]
where $u$ and $v$ are length $n$ vectors, and $D$ is a diagonal matrix whose diagonal entries we will encode in the vector $d,$ \emph{i.e.} $D_{i,i} = d_i.$ (In the course of thinking about this problem, you may stumble across the so-called Sherman-Morrison-Woodbury formula and find that mathematically it provides a potential solution\textemdash you are welcome to try it out and see what happens. However, it will yield an even worse solution, both in terms of accuracy and the residual, for the problem instance in part (c) than working off of an $LU$ factorization (even though we have omitted pivoting in this case).
\newline
 
\emph{ (a) If we assume that we may compute the $LU$ decomposition of $A$ without any pivoting (\textbf{note that in general this is not a good idea, but for the purpose of this question we will make such an assumption}), devise a means to solve a linear system of the form $Ax=b$ using $\mathcal{O}(n)$ time. Here we will also consider the cost of memory allocation and therefore you cannot form $A$ explicitly as that would take $\mathcal{O}(n^2)$ time.} \newline \newline 
Here, one first seeks 2 $n \times n$ matrices $L$ and $U$ such that $A = LU$, for the given matrix $A$. If one expands $A$, the following is obtained,  

$$ 
 A = D + ue_n^T + e_nv^{T} =   
 \begin{bmatrix}
    d_{11}  &  0 & \dots  \\
     0   &\ddots  & 0 & \dots  \\ 
    \vdots   & 0  &   & \dots & d_{nn}  
\end{bmatrix}  +
\begin{bmatrix}
   0   &  0 & \dots & u_{1}   \\
      \vdots   &0  & \dots  &    \vdots \\ 
     0   &0  & \dots  &    u_{n}  
\end{bmatrix}   + \begin{bmatrix}
   0   &  0 & \dots & 0   \\
      \vdots   &0  & \dots  &    \vdots \\ 
     v_{1}   & v_{i}  & \dots  &    v_{n}  
\end{bmatrix} =  $$ 
$$ 
\begin{bmatrix}
   d_{11}   &  0 & \dots & u_{1}   \\
      0   &    \ddots  & \dots  &    u_{i} \\ 
        \vdots   &    0  & d_{ii}  &    \vdots \\ 
     v_{1}   & v_{i}  & \dots  &    (v_{n} + u_{n} + d_{nn}  )
\end{bmatrix} 
$$ 
\newline
Hence, one seeks $L$ and $U$ without pivoting, such that $LU$ yields the above matrix. Let's start with finding $L$, 

$$ L = \begin{bmatrix}
   1   &  0 & \dots & 0  \\
      0   &    \ddots  & \dots  &   \vdots\\ 
        \vdots   &    0  &1  &    \\ 
v_{1}/d_{11}   & \dots  &  v_{n-1}/d_{n-1, n-1}   & 1 
\end{bmatrix} $$ 
This matrix can be computed straightforwardly, since it is known that all diagonal entries have value of 1, while every other entry except row $n$ takes a value of 0. Hence, one only needs to determine the elements of the $n$th row. 
Similarly, the matrix $U$ would take the form, 
$$ U = \begin{bmatrix}
   d_{11}   &  0 & \dots & u_{1}  \\
      0   &    \ddots  & \dots  &   u_{i}\\ 
        \vdots   &    0  &d_{ii}  &  \vdots  \\ 
0   & \dots  &   0   & (d_{nn} + u_{n} + v_{n}) - \sum_{i=1}^{n-1} \frac{v_{i}u_{i}}{d_{ii}}
\end{bmatrix} $$ 
Now, the challenge remains in solving $Ax = b$ for vectors $x, b$ of dimension $n \times 1$. Using the $A= LU$ decomposition above,  one can solve the expressions $Ly = b$ and $Ux = y$ separately. Each of these operations can be done in  $\mathcal{O}(n)$ time. Now, solving for $y$ one obtains $y_{i} = b_{i}$ for $i \leq n-1$. $y_{n}$ would be computed as follows \newline 
$$ y_{n} = b_{n} - \sum_{i=1}^{n-1} \frac{y_{i}v_{i}}{d_{ii}} $$ 
Hence, \newline
$$ y = \begin{bmatrix}
   b_{i}   \\
 \vdots \\ 
 b_{n} - \sum_{i=1}^{n-1} \frac{y_{i} v_{i} }{d_{ii}} 
\end{bmatrix}
$$ 
This algorithm would take  $\mathcal{O}(n)$ time, for $b_{i}$ values are simply being assigned to $y_{i}$ (linear time) for values $i < n$ and computing $b_{n}$ would take $\mathcal{O}(n)$ time since a summation is performed over $n-1$ elements. Then, the overall complexity would be $\mathcal{O}(n)$. \newline 
Now, using this computed $y$ vector, the solution to $Ux=y$ can be determined. The last element of $x$, $x_{n}$ can be determined as follows, 
$$ x_{n} = \frac{  b_{n} - \sum_{i=1}^{n-1} \frac{y_{i} v_{i} }{d_{ii}} }{ (d_{nn} + u_{n} + v_{n}) - \sum_{i=1}^{n-1} \frac{v_{i}u_{i}}{d_{ii}}} $$
Similarly, for $i \leq n-1$ one would seek a solution to (using previously determined $x_{n}$) , 
$$\sum_{i=1}^{n-1} d_{i}x_{i}+ x_{n}u_{i} = y $$ 
 This can be solved as follows, 
 $$x = \sum_{i=1}^{n-1} \frac{y_{i} - x_{n}u_{i}}{  d_{ii}} $$
This is simply a sum over all $i$ values, so it would would take $\mathcal{O}(n)$ time.  This concludes the solution to $x$ in the  system $Ax = b$ for this particular $A$. \newline \newline 
\emph{(b) Implement your method and demonstrate that it achieves the desired scaling.} 
\newline 
Below is the implemented algorithm in Matlab, for the defined matrix $A$ and vector, $b$. 
\begin{lstlisting}
A= [6 7 8; 1 3 5; 13 17 12]; %test A 
b1=[52, 21, 33]; %test b 
n = length(b1); %len of b 
alpha= A(n,n); %%d(n, n) + u(n) + v(n) 
x=zeros(n); 
y=zeros(n); 
u1 = A(1:n-1, n); 
v1 = A(n, 1:n-1); 
d1=diag(A); 
d2= d1(1:end-1); 
s1 = zeros(1,5);
temp=0;
for k= 1:n-1  
  temp= temp + (b1(k)*v1(k))/d1(k); 
  y(k) = b1(k); 
end  
y(n) = b1(n)-temp;  
temp2=0; 
for k=1:n-1 
    temp2 = temp2 + (v1(k)*u1(k))/d(k); 
end
x(n) = y(n)/(alpha-temp2);
for k=1:n-1 
x(k) = x(k)+((y(k)-x(n)*u1(k))/d2(k)); 
end 
\end{lstlisting}
The attached plot presents the desired linear scaling. 
\newline 
(c) Attached is the error plot. 


\emph{(d) By avoiding pivoting, we were able to get a fast, $\mathcal{O}(n)$ algorithm. However, what do you observe about the accuracy of the solution and how might you explain it? } 
The accuracy of the solution decreases with pivoting.
\newline 





\subsection*{Question 3:}
Consider the following $n \times n$ matrix
\[
A = \begin{bmatrix}
1 & 0 & 0 & \cdots & 0 & 1 \\
-1 & 1 & 0 & \cdots & 0 & 1 \\
\vdots & \ddots & \ddots & \ddots & \vdots & \vdots \\
-1  & \cdots & -1 & 1 & 0  & 1 \\
-1  & \cdots & -1 & -1 & 1  & 1 \\
-1  & \cdots & -1 & -1 & -1  & 1 \\
\end{bmatrix}
\]
that has all ones one the diagonal and in the last column, and all the entries below the diagonal are $-1.$ \newline 
\emph{ (1) Consider computing an $LU$ decomposition of $A$ (with or without partial pivoting, the answer will be the same). Work out an expression for the largest entry of $U$ in terms of $n.$ }    
\newline
\newline 
One can compute the $LU$ decomposition for $n=1-4$ and observe the trend, particularly the factor $U$. Below are the decompositions for $n=2 -4$,  (computed, via the lu routine in Matlab)
$$  U_{2} = \begin{bmatrix}
1 & 1 \\
0 & 2 
\end{bmatrix}, 
U_{3} = 
\begin{bmatrix}
1 & 0 &1 \\
0 & 1 & 2 \\ 
 0 & 1 & 4  
\end{bmatrix},  
U_{4} = 
\begin{bmatrix}
1 & 0 & 0 & 1 \\
0 & 1 & 0 & 2 \\ 
 0 & 0 & 1 & 4 \\ 
  0 & 0 & 0 & 8 
\end{bmatrix}, 
$$  
The trend is clear that the largest value in $U$ for each of these matrices scales with $n$ by the expression, $b = 2^{n-1}$. This is the relationship between matrix size and the largest entry in the upper triangular matrix, $U$ for $A = LU$. 
\newline \newline 
\emph{(b) If we computed an $LU$ factorization in practice and encountered an entry of this size do you think it would be problematic? why or why not?}  \newline   
This would be problematic if we encountered a matrix of a very large size. We define the machine precision of the LU decomposition as, 
$$ \mathcal{O}(\mu) = \frac{||\delta A||_{2}}{||L||_{2}||U||_{2}}$$ 
Hence, if we encountered larger matrices $A$, we would encounter an exponentially larger values norms of $U$, as shown in part (a). Effectively, this would yield an exponential decrease in precision (exponential increase in error) as described by the above equation. Hence, this becomes very problematic. \newline \newline 
\emph{(c) Can you think of a different style of pivoting that would avoid the issues highlighted here? Briefly outline/describe your proposed procedure.} \newline \newline  
A different style of pivoting in this case would be designed in a manner, such that we avoid the generation of large entries in the last column of $U$. In the partial pivoting used in the LU decomposition, one encounters the swapping of rows (due to the permutation matrix), such that the last row of U is larger than the first its preceding row. Instead, one should design an algorithm such that the rows are swapped for rows with large entries of $A$ are in lower order rows of $A$. Hence when pivoting, the largest values of $A$ should  be at the in the first rows of $A$. Effectivley, this would avoid the generation of large non-zero entries accumulating in the last column of $U$ in the decomposition. This would then avoid the error issue, having a smaller norm of $U$ and hence increase the machine precision. 



\end{document}